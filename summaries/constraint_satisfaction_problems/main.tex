\documentclass{article}
\usepackage{graphicx} % Required for inserting images

\usepackage{algorithm2e}

\usepackage{amsmath}

\title{Constraint Satisfaction Problems}
\author{Daniel Gigliotti}
\date{}

\begin{document}

\maketitle

\section{Constraint Satisfaction Problems}

Constraint Satisfaction Problems (CSPs) involve finding a solution that satisfies a set of constraints or conditions. These problems are commonly used in various domains, including scheduling, planning, optimization, and decision-making.

\begin{center}
    Constraint Satisfaction Problem: find an \textbf{assignment} \\ of values for each variable from its respective domain \\ such that every constraint is satisfied.
\end{center}

Constraint Satisfaction Problems consist of:
\begin{itemize}
    \item A set of \textbf{variables}
    \begin{equation*}
        V = \{v_1, v_2, ..., v_n\}
    \end{equation*}
    \item A set of \textbf{domains}, one for each variable
    \begin{equation*}
        D = \{D_{v_1}, D_{v_2}, ..., D_{v_n}\}
    \end{equation*}
    \item A set of \textbf{constraints} (binary relations), conditions that the solution must satisfy
    \begin{equation*}
        C = \{C_{\{u, v\}}\}
    \end{equation*}
    \begin{equation*}
        u, v \in V, u \neq v, C_{\{u, v\}} \in C
    \end{equation*}
\end{itemize}

We can visualize a CSP with a \textbf{constraint graph}, that is an undirected graph with variables as vertices and constraints as edges (the graph has an arc $\{u, v\}$ iff $C_uv \in C$. \\

A \textbf{partial assignment} assigns \textbf{some} variables to values from their respective domains. \\
A \textbf{total assignment} is an assignment defined on all the variables. \\
We say that a partial assignment is \textbf{consistent} if it does not violate any constraint.\\
A total assignment that is consistent is a \textbf{solution} for the problem.\\
A problem $\gamma$ is \textbf{solvable} if there is a total consistent assignment $\alpha$ that is a solution for $\gamma$; if such solution does not exists, the problem is \textbf{inconsistent}.\\
Let $\alpha$ be a partial assignment; we say that $\alpha$ can be \textbf{extended to a solution} if there exists a solution $\alpha'$ that agrees with $\alpha$ on the variables where $\alpha$ is defined. \\

An algorithm is \textbf{complete} if it terminates with a solution when one exists.
An algorithm is \textbf{sound} if it does not yield any results that are untrue.

\section{Backtracking}

Backtracking is a general algorithmic technique used to solve constraint satisfaction problems very efficiently. It is much better than enumerating and solution-checking all total assignments because a partial assignment that is already inconsistent is excluded and no longer clutters when computing the solution. \\

The naive form of backtracking is the following:



\end{document}
