\documentclass{article}
\usepackage{graphicx} % Required for inserting images

\usepackage{listings}

\usepackage{amsmath}

\title{Constraint Satisfaction Problems}
\author{Daniel Gigliotti}
\date{}

\begin{document}

\maketitle

\section{Constraint Satisfaction Problems}

Constraint Satisfaction Problems (CSPs) involve finding a solution that satisfies a set of constraints or conditions. These problems are commonly used in various domains, including scheduling, planning, optimization, and decision-making.

\begin{center}
    Constraint Satisfaction Problem: find an \textbf{assignment} \\ of values for each variable from its respective domain \\ such that every constraint is satisfied.
\end{center}

Constraint Satisfaction Problems consist of:
\begin{itemize}
    \item A set of \textbf{variables}
    \begin{equation*}
        V = \{v_1, v_2, ..., v_n\}
    \end{equation*}
    \item A set of \textbf{domains}, one for each variable
    \begin{equation*}
        D = \{D_{v_1}, D_{v_2}, ..., D_{v_n}\}
    \end{equation*}
    \item A set of \textbf{constraints} (binary relations), conditions that the solution must satisfy
    \begin{equation*}
        C = \{C_{\{u, v\}}\}
    \end{equation*}
    \begin{equation*}
        u, v \in V, u \neq v, C_{\{u, v\}} \in C
    \end{equation*}
\end{itemize}

We can visualize a CSP with a \textbf{constraint graph}, that is an undirected graph with variables as vertices and constraints as edges (the graph has an arc $\{u, v\}$ iff $C_uv \in C$. \\

A \textbf{partial assignment} assigns \textbf{some} variables to values from their respective domains. \\
A \textbf{total assignment} is an assignment defined on all the variables. \\
We say that a partial assignment is \textbf{consistent} if it does not violate any constraint.\\
A total assignment that is consistent is a \textbf{solution} for the problem.\\
A problem $\gamma$ is \textbf{solvable} if there is a total consistent assignment $\alpha$ that is a solution for $\gamma$; if such solution does not exists, the problem is \textbf{inconsistent}.\\
Let $\alpha$ be a partial assignment; we say that $\alpha$ can be \textbf{extended to a solution} if there exists a solution $\alpha'$ that agrees with $\alpha$ on the variables where $\alpha$ is defined. \\

An algorithm is \textbf{complete} if it terminates with a solution when one exists.
An algorithm is \textbf{sound} if it does not yield any results that are untrue.

\section{Backtracking}

Backtracking is a general algorithmic technique used to solve constraint satisfaction problems very efficiently. It is much better than enumerating and solution-checking all total assignments because a partial assignment that is already inconsistent is excluded and no longer clutters when computing the solution. \\

\begin{flushleft}
    The naive form of backtracking is the following:
\end{flushleft}

\begin{lstlisting}[language=Python, mathescape=true, escapeinside={(*}{*)}]
def naive_backtracking($\gamma$, $\alpha$):

    if (*$\alpha$ is inconsistent with $\gamma$:*)
        return "inconsistent"

    if (*$\alpha$ is a total assignment:*)
        return $\alpha$

    (*select \textbf{some} variable $v$ for which $\alpha$ is not defined*)
    (*for each $d \in D_v$ \textbf{in some order} do:*)
        (*$\alpha' := \alpha \cup \{v=d\}$ (\textbf{instantiate the variable $v$ with $d$})*)
        (*$\alpha'' := $ naive\_backtracking($\gamma, \alpha' $)*)
        if (*$\alpha''$ is not inconsistent:*)
            return $\alpha''$

    return "inconsistent"
    
\end{lstlisting}

Backtracking recursively instantiate variables one by one, backing up out of a search branch if the current partial assignment is already inconsistent. Backtracking does not recognize $\alpha$ that cannot be extended to a solution, unless $\alpha$ is already inconsistent. We can employ \textbf{inference} for this and speed up the algorithm. Other potential improvement points are highlighted in bold.\\

Backtracking is \textbf{complete}: it will find the solution.

\newpage

\begin{itemize}
    \item A widely used \textbf{variable ordering strategy} is to \textbf{choose the most constrained variable first}: this way we reduce the branching factor (number of sub-trees generated for the variable) and thus reduce the size of the search tree.
    \item A widely used \textbf{value ordering strategy} is to \textbf{choose the least constraining value first}: this way we increase the chances not to rule out the solutions below the current node. 
\end{itemize}

\newpage

\section{Inference}

The backtracking algorithm can be extended with inference techniques. Inference reduces the search space. 

\begin{center}
    \textbf{Inference}: adding constraints without losing solutions; \\ we obtain an equivalent network with a tighter description \\ and hence a smaller number of consistent partial assignments.
\end{center}

We deduce additional constraints (unary or binary) that follows from the already known constraints and hence are satisfied in all solutions. \\

We have two inference methods:
\begin{itemize}
    \item Forward checking;
    \item Arc consistency;
\end{itemize}

Two networks $\gamma$ and $\gamma'$ are equivalent if $\gamma$ has the same solutions as $\gamma'$. A network $\gamma$ is tighter than $\gamma'$ if $\gamma$ has the same constraints as $\gamma'$ plus some others.

\begin{itemize}
    \item $\gamma \subset \gamma' \rightarrow \gamma$ is strictly tighter than $\gamma'$
    \item $\gamma \subseteq \gamma' \rightarrow \gamma$ is tighter than $\gamma'$
\end{itemize}

\textbf{Equivalence + tightness = inference}. Let $\gamma$ and $\gamma'$ be constraint networks such that $\gamma \equiv \gamma'$ ($\gamma$ is equivalent to $\gamma'$) and $\gamma' \subset \gamma$ ($\gamma'$ is tighter than $\gamma$). Then $\gamma'$ has the same solutions as $\gamma$ (definition of equivalence), but less consistent partial assignments than $\gamma$:
\begin{center}
    $\gamma'$ is a better encoding of the underlying problem.
\end{center}

Given a total assignment $\alpha$:
\begin{itemize}
    \item if $\alpha$ is consistent with $\gamma$, it could be inconsistent with $\gamma'$.
    \item if $\alpha$ cannot be extended to a solution in $\gamma$, the same thing holds for $\gamma'$ ($\gamma \equiv \gamma'$).
\end{itemize}

\newpage

\begin{lstlisting}[language=Python, mathescape=true, escapeinside={(*}{*)}]
def backtracking_with_inference($\gamma$, $\alpha$):

    if (*$\alpha$ is inconsistent with $\gamma$:*)
        return "inconsistent"

    if (*$\alpha$ is a total assignment:*)
        return $\alpha$

    # inference
    (*\textbf{$\gamma' :=$ a copy of $\gamma$}*)
    (*\textbf{$\gamma' :=$ inference($\gamma'$)}*)
    (*\textbf{if exists $v$ with $D_v'=\theta$:}*)
        (*\textbf{return "inconsistent"}*)

    (*\textbf{select the most constrained variable $v$ for which $\alpha$ is not defined}*)
    (*\textbf{for each $d \in D_v$ (least constraining value)}:*)
        (*$\alpha' := \alpha \cup \{v=d\}$*)
        (*$\alpha'' := $ backtracking\_with\_inference($\gamma, \alpha' $)*)
        if (*$\alpha''$ is not inconsistent:*)
            return $\alpha''$

    return "inconsistent"
    
\end{lstlisting}

Inference($\gamma$) is any procedure that delivers a tighter equivalent network.
\begin{itemize}
    \item \textbf{forward checking}: from assigned to unassigned variables; after an assignment to a variable, look at the constraints of that variable, take the other end of each constraint (since it is a binary constraint) and remove the values of that end that would not comply with the assignment just made.
    \item \textbf{arc consistency}. \\
    \begin{center}
        \textbf{Consistency}: for every constraint and every domain value, \\ at least one value on the other side of the constraint \\ needs to make the constraint valid.
    \end{center}
    \begin{itemize}
        \item \textbf{Revise($\gamma, u, v$)} is an algorithm enforcing consistency of $u$ relative to $v$ ($O(k^2)$), for just a pair of variables.
        \item \textbf{AC-1}: apply Revise($\gamma, u, v$) up to a fixed point with redundant computations.
        \item \textbf{AC-3}: apply Revise($\gamma, u, v$) up to a fixed point remembering the potentially inconsistent variable pairs.
    \end{itemize}
\end{itemize}

\newpage

\section*{Forward Checking}
\begin{lstlisting}[language=Python, mathescape=true, escapeinside={(*}{*)}]
def ForwardChecking($\gamma$, $a$):

    (*for each $v$ where $a(v) = d'$ is defined do:*)
        (*for each $u$ where $a(u)$ is undefined and $C_{u,v} \in C$ do:*)
        
            # only leave valid values in $D_u$, values that satisfy 
            # the constraints of $u$ with $v$
            $D_u := \{d | d \in D_u, (d, d') \in C_{u,v}\}$

    return $\gamma$

\end{lstlisting}

\section*{Revise}
\begin{lstlisting}[language=Python, mathescape=true, escapeinside={(*}{*)}]
def Revise($\gamma$, $u$, $v$):

    (*for each $d \in D_u$:*)
        (*if there is no value $\in D_v$ for which $C_{u, v}$ is satisfied:*)
            # remove $d$ from $D_u$
            $D_u := D_u / \{d\}$

    return $\gamma$

\end{lstlisting}

\section*{AC-1}
\begin{lstlisting}[language=Python, mathescape=true, escapeinside={(*}{*)}]
def AC1($\gamma$):

    changes_made = False
    while !changes_made:

        for each constraint $C_{u, v}$:
        
            $\gamma' = Revise(\gamma, u, v)$
            if $D_u$ reduces:
                changes_made = True
            
            $\gamma' = Revise(\gamma, v, u)$
            if $D_v$ reduces:
                changes_made = True

    return $\gamma$

\end{lstlisting}

\section*{AC-3}
\begin{lstlisting}[language=Python, mathescape=true, escapeinside={(*}{*)}]
def AC3($\gamma$):

    $M := \emptyset$
    (*for each constraint $C_{u,v} \in C$ do:*)
        $M := M \cup \{(u,v), (v,u)\}$

    (*while $M \neq \emptyset$ do:*)
        (*remove any element $(u, v)$ from $M$*)
        $Revise(\gamma, u, v)$
        (*if $D_u$ has changed in the call to Revise, then:*)
            (*for each constraint $C_{w, u} \in C$ where $w \neq u$ do:*)
                $M := M \cup \{(w,u)\}$

    return $\gamma$

\end{lstlisting}

\section*{AcyclicCG}
\begin{lstlisting}[language=Python, mathescape=true, escapeinside={(*}{*)}]
def AcyclicCG($\gamma$):

    # get a directed tree from $\gamma$ picking an arbitrary
    # variable $v$ as root and directing arcs outwards
    $\gamma' \leftarrow\ GetDirectedTree(\gamma, v)$

    # order the variables topologically i.e. such that each
    # vertex is ordered before its children; denote that 
    # by $v_1, v_2, ..., v_n$
    vertex_list = $Sort(\gamma'$)

    # for $i := n, n - 1, ..., 2$
    for i in range(len(vertex_list), 2, -1):
        $Revise(\gamma, v_{parent(i)}, v_i)$
        if $D_p = \emptyset$:
            return "inconsistent"

    # now every variable is arc consistent relative to its children.
    # Run backtracking with inference with forward checking using the 
    # variable order $v_1, v_2, ..., v_n$
    $Backtrack(vertex\_list)$

\end{lstlisting}

\end{document}
