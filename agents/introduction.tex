\documentclass{article}
\usepackage[utf8]{inputenc}

% images
\usepackage{graphicx}
\graphicspath{ {./images/} }

\title{agents}
\author{Daniel Gigliotti}
\date{}

\begin{document}

\maketitle

\section{Rational agents}

An agent is an entity that perceives and acts; a rational agent, on the other hand, is a person or entity that always aims to perform optimal actions based on given premises and information.

\section{Agent and environment}

An AI system can be defined as the study of a rational agent and its environment. 
A rational agent can be anything that makes decisions, typically a person, robot or software. An agent can be seen as a black box interacting with the environment in two ways:

\begin{center}
    \begin{itemize}
        \item takes input from the environment (perceives)
        \item produces an output (acts)
    \end{itemize}
\end{center}

The agents sense the environment through sensors and act on their environment through actuators.

How can we implement this box? How do the system will convert perceptions into actions?

\section{Agent architecture}

Abstractly, an agent can be seen as a function that maps from percept histories to actions:

\begin{center}
    \begin{equation}
        f: P^{*} \rightarrow A
    \end{equation}
\end{center}

A $percept$ is a sequence of $perceptions$: maybe previous perceptions could help us in the future.

For a given task we seek the agent (or class of agents) with the best performance: it aims to choose whichever action maximizes the expected value of the performance measure given the percept sequence to date.
Computational limitations make perfect rationality unachievable.

\section{Agent types}

Four basic types in order of increasing generality:

\begin{center}
    \begin{itemize}
        \item simple reflex agents
        \item model-based reflex agents
        \item goal-based agents
        \item utility-based agents
    \end{itemize}
\end{center}

All these can be turned into learning agents. A learning agent is an agent with the capability of learning from its previous experience. What is it needed to talk about learning agents?

\begin{itemize}
    \item $Learning\ element$: element that enables learning from previous experience;
    \item $Critic$: provides feedback on how well the agent is doing concerning a fixed performance standard;
    \item $Performance\ element$: the actions to be performed are selected;
    \item $Problem\ generator$: acts as a feedback agent that performs certain tasks such as making suggestions that will lead to new and informative experiences;
\end{itemize}

\end{document}